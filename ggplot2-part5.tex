Chapter 5 Toolbox
5.1 Introduction
5.2 Overall Layering Strategy
5.3 Basic Plot Types
5.4 Displaying Distributions
5.5 Dealing with over-plotting
5.6 Surface Plots
5.7 Drawing Maps
5.8 Revealing Uncertainty
5.9 Statistical Summaries
5.10 Annotating a plot
%-----------------------------------------------------------------------------%
\subsection*{5.1 Introduction}
The layered structure of ggplot2 encourages design and construction of graphics in a structured manner.
%-----------------------------------------------------------------------------%
\subsection*{5.3 Basic Plots}
Geoms are the fundamental building blocks of ggplot2, useful in their own right, and in conjunction to make more complex plots.

geom_area : draws an area plot
geom_bar : barchart
geom_line : produces a line plot
geom_point: produces a scatterplot.
geom_polygon: draws polygons, which are filled paths.
geom_text :  add labels at the specified points
geom_tile :  makes an image plot or level plot.
%-----------------------------------------------------------------------------%
\subsection*{5.4 Displaying distributions}
There are a number of geoms that can be used to display distributions, depending on 
the dimensionality of the distribution
whether the data is continuous or discrete
whether it is conditional or joint distribution is of interest

%-----------------------------------------------------------------------------%
5.4
\begin{verbatim}
qplot(class,cty,data=mpg,geom=”jitter”)
qplot(class,drv,data=mpg,geom=”jitter”)

geom_jitter = position_jitter + geom_point
\end{verbatim}

A crude way of looking at discrete distributions by adding some jitter to prevent overplotting.
geom_density = stat_density + geom_area :
A smoothed version of the frequency polygon based on kernel smoothers.

\begin{verbatim}
qplot(depth,data=diamonds,geom=”density”,xlim=c(54,70))

qplot(depth,data=diamonds,geom=”density”,xlim=c(54,70), 
     fill =cut, alpha=I(1/5))
\end{verbatim}     
%-----------------------------------------------------------------------------%
\subsection*{5.5 dealing with overplotting}
Scatterplot is a very important tool for assessing the relationship between two continuous variables.

If the dataset is large, often points will be plotted on top of each other, obscuring the true relationship.
%-----------------------------------------------------------------------------%
5.9

5.9.1 Individual summary functions
5.9.2 single summary fuction
